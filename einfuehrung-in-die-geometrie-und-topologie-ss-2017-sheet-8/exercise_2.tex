\section{}





\subsection{}

Für alle $A, B \in \matrices{n}{k}$ ist die Abbildung $[0,1] \to \matrices{n}{k}$, $t \mapsto (1-t)A + tB$ ein stetiger Weg von $A$ nach $B$.





\subsection{}


\begin{proposition}
  \label{proposition: elemantery row operations}
  Es sei $k \in \{\real, \complex\}$ und $A \in \GL{n}{k}$.
  Die Matrix $B \in \GL{n}{k}$ entstehe aus $A$ durch
  \begin{enumerate}[label = \alph*)]
    \item
      \label{operation: adding multiples to other rows}
      Addition des $\lambda$-fachen der $i$-ten Zeile auf die $j$-te Spalte für $i \neq j$ und $\lambda \in k$;
    \item
      \label{operation: scaling rows}
      Multiplikation der $i$-ten Zeile mit $\lambda \in k \setminus \{0\}$, wobei $\lambda$ in der gleichen Wegzusammenhangskomponente liegt wie $1 \in k \setminus \{0\}$
      (also $\lambda \neq 0$ für $k = \complex$ und $\lambda > 0$ für $k = \real$);
  \end{enumerate}
  oder
  \begin{enumerate}[label = \alph*), resume]
    \item
      \label{operation: swapping rows}
      im Fall $k = \complex$ durch Vertauschen der $i$-ten und $j$-ten Zeile für $i \neq j$;
    \item
      \label{operation: modified swapping rows}
      im Fall $k = \real$ durch Vertauschen der $i$-ten und $j$-ten Zeile für $i < j$ und anschließendes Multiplizieren der $i$-ten Zeile (d.h.\ der früheren $j$-ten Zeile) mit $-1$.
  \end{enumerate}
  Dann gibt es in $\GL{n}{k}$ einen stetigen Weg von $A$ nach $B$.
\end{proposition}

\begin{proof}
  \begin{enumerate}[label = \alph*)]
    \item
      Für alle $\mu \in K$ sei $E(\mu) \in \matrices{n}{k}$ die Matrix der Form
      \[
                  E(\mu)
        \coloneqq \begin{pmatrix}
                    1 &         &         &   \\
                      & \ddots  & \mu     &   \\
                      &         & \ddots  &   \\
                      &         &         & 1
                  \end{pmatrix},
      \]
      wobei $\mu$ im $(j,i)$-ten Eintrag steht.
      Für alle $C \in \matrices{n}{k}$ entsteht die Matrix $E(\mu) C$ aus $C$ durch Addition des $\mu$-fachen der $i$-ten Zeile auf die $j$-te.
      Dann ist
      \[
                \alpha
        \colon  [0,1] \to \GL{n}{k},
        \quad   t
        \mapsto E(t\lambda) A
      \]
      ein stetiger Weg von $A$ nach $E(\lambda) A = B$.
      Die Wohldefiniertheit von $\alpha$ folgt daraus, dass $\det \alpha(t) = \det A$ für alle $t \in [0,1]$ gilt.
      Die Stetigkeit von $\alpha$ folgt daraus, dass die Einträge von $\alpha(t)$ affin-lineare Funktionen in $t$ sind.
      
    \item
      Für alle $\mu \in k \setminus \{0\}$ sei $E(\mu) \in \matrices{n}{k}$ die Diagonalmatrix mit Einträgen $E(\mu)_{ii} = \mu$ und $E(\mu)_{jj} = 1$ für alle $j \neq i$.
      Für alle $C \in \matrices{n}{k}$ entsteht die Matrix $E(\mu) C$ aus $C$ durch Multiplikation der $i$-ten Zeile mit $\mu$.
      
      Nach Annahme gibt es einen stetigen Weg $\alpha \colon [0,1] \to k \setminus \{0\}$ von $1$ nach $\lambda$.
      Damit ist
      \[
                \beta
        \colon  [0,1] \to \GL{n}{k},
        \quad   t
        \mapsto E(\alpha(t)) A
      \]
      ein stetiger Weg von $A$ nach $E(\alpha(1)) A = E(\lambda) A = B$.
      Die Wohldefiniertheit von $\beta$ folgt daraus, dass $\det \beta(t) = \alpha(t) \det A \neq 0$ für alle $t \in [0,1]$ gilt.
      Die Stetigkeit von $\beta$ folgt daraus, dass die Einträge von $\beta(t)$ affin-linerae Funktionen in $t$ sind.
      
      Nach Lemma~\ref{lemma: path components of punctered euclidian space} ist $\real_+ = \{x \in \real \suchthat x > 0\}$ die Wegzusammenhangskomponente von $1$ in $\real \setminus \{0\}$, und $\complex \setminus \{0\}$ die Wegzusammenhangskomponente von $1$ in $\complex \setminus \{0\}$.
    
    \item
      Wir können ausgehend von $A$ die folgenden elementaren Zeilenoperationen durchführen, ohne die Wegzusammenhangskomponente von $A$ zu verlassen:
      \begin{enumerate}[label = \arabic*.]
        \item
          Addition der $i$-ten Zeile auf die $j$-te Zeile.
        \item
          Subtraktion der $j$-ten Zeile von der $i$-ten Zeile.
        \item
          Addition der $i$-ten Zeile auf die $j$-te Zeile.
        \item
          Multiplikation der $i$-ten Zeile mit $-1$.
      \end{enumerate}
      Hierdurch entsteht genau die Matrix $B$.
      Die Änderung der $i$-ten und $j$-ten Zeile lässt sich dabei wie folgt darstellen:
      \[
        \begin{matrix}
          z_i \\ z_j
        \end{matrix}
        \;\to\;
        \begin{matrix}
          z_i \phantom{+ z_j} \\ z_i + z_j
        \end{matrix}
        \;\to\;
        \begin{matrix}
          \phantom{z_i} - z_j \\ z_i + z_j
        \end{matrix}
        \;\to\;
        \begin{matrix}
          -z_j \\ \phantom{-}z_i
        \end{matrix}
        \;\to\;
        \begin{matrix}
          z_j \\ z_i
        \end{matrix}
      \]
      
    \item
      Es muss im obigen Verfahren nur der vierte Schritt weggelassen werden.
    \qedhere
  \end{enumerate}
\end{proof}





\subsubsection{Der komplexe Fall \texorpdfstring{$\GL{n}{\complex}$}{GLn(C)}}

Es sei $A \in \GL{n}{\complex}$.
Proposition~\ref{proposition: elemantery row operations} zeigt, dass man ausgehend von $A$ alle elementaren Zeilenoperationen durchführen darf, ohne die Wegzusammenhangskomponente von $A$ zu verlassen.
Da $A$ invertierbar ist, kann man $A$ durch elementare Zeilenumformungen in die Einheitsmatrix umformen.

Es folgt, dass jede Wegzusammenhangskomponente von $\GL{n}{\complex}$ die Einheitsmatrix enthält.
Somit ist $\GL{n}{\complex}$ wegzusammenhängend.





\subsubsection{Der reelle Fall \texorpdfstring{$\GL{n}{\real}$}{GLnR}}

Wir zeigen, dass $\GL{n}{\real}$ in die beiden (Weg)zusammenhangskomponenten
\[
            \GL{n}{\real}_+
  \coloneqq \{A \in \GL{n}{\real} \suchthat \det A > 0\}
  \quad\text{und}\quad
            \GL{n}{\real}_-
  \coloneqq \{A \in \GL{n}{\real} \suchthat \det A < 0\}
\]
zerfällt.

Wäre $\GL{n}{\real}$ zusammenhängend, so wäre auch das Bild $\det( \GL{n}{\real} ) = \real \setminus \{0\}$ zusammenhängend.
Nach Lemma~\ref{lemma: path components of punctered euclidian space} ist $\real \setminus \{0\}$ aber unzusammenhängend.
Folglich besteht $\GL{n}{\real}$ aus mindestens zwei Zusammenhangskomponenten.
Es genügt daher zu zeigen, dass $\GL{n}{\real}_+$ und $\GL{n}{\real}_-$ jeweils wegzusammenhängend sind.

Wir zeigen zunächst, dass sich jede Matrix $A \in \GL{n}{\real}$ durch Operationen, welche jeweils die Wegzusammenhangskomponente von $A$ nicht verlassen, in eine der beiden Diagonalmatrizen
\[
            D_+
  \coloneqq \begin{pmatrix}
              1 &         &   &   \\
                & \ddots  &   &   \\
                &         & 1 &   \\
                &         &   & 1
            \end{pmatrix}
            \in \GL{n}{\real}_+
  \quad\text{und}\quad
            D_-
  \coloneqq \begin{pmatrix}
              1 &         &   &     \\
                & \ddots  &   &     \\
                &         & 1 &     \\
                &         &   & -1
            \end{pmatrix}
            \in \GL{n}{\real}_-
\]
umformen lässt:

\begin{itemize}
  \item
    Da $A$ invertierbar ist, können wir $A$ durch elementare Zeilenumformungen in die Einheitsmatrix umformen.
    
    Schränkt man dabei das Skalieren von Zeilen auf positive Skalare ein, und Ersetzt das Vertauschen von Zeilen (eine Umformungen vom Typ \ref{operation: swapping rows} in Propositon~\ref{proposition: elemantery row operations}) durch die äbgeänderte Operation \ref{operation: modified swapping rows} aus Propositon~\ref{proposition: elemantery row operations}, so kann $A$ mit quasi unveränderten Vorgehen in ein Matrix der Form
    \[
        D
      = \begin{pmatrix}
          \varepsilon_1 &         &               \\
                        & \ddots  &               \\
                        &         & \varepsilon_n
        \end{pmatrix}
      \qquad
      \text{mit $\varepsilon_i = \pm 1$ für alle $i = 1, \dotsc, n$}
    \]
    umformt werden, ohne die Wegzusammenhangskomponente von $A$ zu verlassen.
    
  \item
    Wendet man für $i < j$ die Operation \ref{operation: modified swapping rows} zweimal hintereinander an, so ändert man ingesamt das Vorzeichen der $i$-ten und $j$-ten Zeile:
    \[
      \begin{matrix}
        z_i \\ z_j
      \end{matrix}
      \;\to\;
      \begin{matrix}
        -z_j \\ \phantom{-}z_i
      \end{matrix}
      \;\to\;
      \begin{matrix}
        -z_i \\ -z_j
      \end{matrix}
    \]
    Durch iteriertes Anwenden hiervon lässt sich $D$ in eine der Formen $D_+$ oder $D_-$ bringen, ohne die Wegzusammenhangskomponente von $A$ zu verlassen:
    
    Falls $|\{1 \leq i \leq n \suchthat \varepsilon_i = -1\}|$ gerade ist, so kann man die Minus-Einsen auf der Diagonale von $D$ paarweise durch Einsen ersetzen, bis man schließlich $D_+$ erhält.
    Falls $|\{1 \leq i \leq n \suchthat \varepsilon_i = -1\}|$  ungerade ist, so ist man im Fall $n = 1$ bereit fertig; ansonsten ändert man das Vorzeichen von $\varepsilon_{n-1}$ und $\varepsilon_n$, und elemeniert anschließend die Minus-Einsen im Bereich $1 \leq i \leq n-1$ paarweise.
    Hierdurch erhält man schließlich $D_-$.
\end{itemize}

Man bemerke, dass die Umformungen vom Typ \ref{operation: adding multiples to other rows} und \ref{operation: modified swapping rows}, sowie das obige paarweise Eliminieren von Minus-Einsen  die Determinante einer Matrix nicht verändert, und dass Umformungen vom Typ \ref{operation: scaling rows} das Vorzeichen der Determinante nicht verändert.
Ingesamt bleibt also das Vorzeichen der Determinante bei allen genutzen Umformungen unverändert.

Es folgt, dass sich alle $A \in \GL{n}{\real}_+$ zu $D_1$ umformen lassen, und alle $A \in \GL{n}{\real}_-$ zu $D_{-1}$, jeweils ohne die entsprechende Wegzusammenhangskomponente zu verlassen.
Inbesondere sind $\GL{n}{\real}_+$ und $\GL{n}{\real}_-$ wegzusammenhängend.


% TODO: Remark regarding rotation and reflection matrices.

% TODO: Remark regarding outher matrix groups.










