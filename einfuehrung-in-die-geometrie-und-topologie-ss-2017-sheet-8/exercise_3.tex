\section{}


\begin{lemma}[Verkleben stetiger Abbildungen]
  \label{lemma: glueing lemma for continuous maps}
  Es seien $X$ und $Y$ topologische Räume.
  \begin{enumerate}[label = \alph*)]
    \item
      Es sei $(U_i)_{i \in I}$ eine Familie von offenen Mengen $U_i \subseteq X$ mit $X = \bigcup_{i \in I} U_i$.
      Für jedes $i \in I$ sei $f_i \colon U_i \to Y$ eine stetige Funktion, so dass $f_i|_{U_i \cap U_j} = f_j|_{U_i \cap U_j}$ für alle $i,j \in I$ gilt.
      Dann gibt es eine eindeutige Funktion $f \colon X \to Y$ mit $f|_{U_i} = f_i$ für alle $i \in I$, und $f$ ist stetig.
    \item
      Es seien $C_1, \dotsc, C_n \subseteq X$ abgeschlossene Mengen mit $X = C_1 \cup \dotsb \cup C_n$.
      Für jedes $i = 1, \dotsc, n$ sei $f_i \colon C_i \to Y$ eine stetige Funktion, so dass $f_i|_{C_i \cap C_j} = f_j|_{C_i \cap C_j}$ für alle $i,j = 1, \dotsc, n$ gilt.
      Dann gibt es eine eindeutige Funktion $f \colon X \to Y$ mit $f|_{C_i} = f_i$ für alle $i = 1, \dotsc, n$, und $f$ ist stetig.
  \end{enumerate}
\end{lemma}

\begin{proof}
  Die Eindeutigkeit und Existenz von $f$ ist jeweils eine rein mengentheoretische Aussage;
  $f$ ist gegeben durch $f(x) = f_i(x)$ für alle $i \in I$ und $x \in U_i$, bzw.\ alle $i = 1, \dotsc, n$ und $x \in C_i$.
  Es bleibt zu zeigen, dass $f$ jeweils stetig ist.
  \begin{enumerate}
    \item
      Es sei $U \subseteq Y$ offen.
      Wegen der Stetigkeit der $f_i$ ist $f_i^{-1}(U) \subseteq U_i$ für alle $i \in I$ offen.
      Wegen der Offenheit von $U_i \subseteq X$ ist $f_i^{-1}(U)$ für alle $i \in I$ auch in $X$ offen.
      Deshalb ist $f^{-1}(U) = \bigcup_{i \in I} f_i^{-1}(U)$ als Vereinigung offener Mengen ebenfalls offen.
    
    \item
      Es sei $C \subseteq Y$ abgeschlossen.
      Wegen der Stetigkeit der $f_i$ ist $f_i^{-1}(C) \subseteq C_i$ für alle $i = 1, \dotsc, n$ abgeschlossen.
      Wegen der Abgeschlossenheit von $C_i \subseteq X$ ist $f_i^{-1}(C)$ für alle $i = 1, \dotsc, n$ auch in $X$ abgeschlossen.
      Deshalb ist $f^{-1}(C) = \bigcup_{i=1}^n f_i^{-1}(C)$ als endliche Vereinigung abgeschlossener Mengen ebenfalls abgeschlossen.
    \qedhere
  \end{enumerate}
\end{proof}

Das Diagramm
\[
  \begin{tikzcd}
      A \cap B
      \arrow{r}{j_A}
      \arrow[swap]{d}{j_B}
    & A
      \arrow{d}{i_A}
    \\
      B
      \arrow[swap]{r}{i_B}
    & X
  \end{tikzcd}
\]
kommutiert.
Um zu zeigen, dass es sich um ein Pushout-Diagramm handelt, gilt es zu zeigen, dass es für jeden topologischen Raum $Y$ und je zwei stetige Funktionen $f_A \colon A \to Y$ und $f_B \colon B \to Y$ mit $f_A \circ j_A = f_B \circ j_B$, eine eindeutige stetige Funktion $f \colon X \to Y$ mit $f \circ i_A = f_A$ und $f \circ i_B = f_B$ gibt.

Die Bedingung $f_A \circ j_A = f_B \circ j_B$ ist äquivalent dazu, dass $f_A|_{A \cap B} = f_B|_{A \cap B}$, und die Bedingungen $f \circ i_A = f_A$ und $f \circ i_B = f_B$ sind jeweils äquivalent dazu, dass $f|_A = f_A$ und $f|_B = f_B$.
Die Aussage folgt deshalb unmittelbar aus Lemma~\ref{lemma: glueing lemma for continuous maps}.




