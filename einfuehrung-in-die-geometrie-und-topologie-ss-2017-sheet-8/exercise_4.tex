\section{}

Es seien $W, W_1, W_2 \coloneqq [0,1]^n$.
(Der Buchstabe „$W$“ steht hier für Würfel.)
Es seien
\begin{alignat*}{3}
            \varphi_1
  \colon    W_1
  &\to      W,
  &\quad    (x_1, \dotsc, x_n)
  &\mapsto  \left( x_1, \dotsc, x_{k-1}, \frac{x_k}{2}, \dotsc, x_k, \dotsc, x_n \right)
\shortintertext{und}
            \varphi_2
  \colon    W_2
  &\to      W,
  &\quad    (x_1, \dotsc, x_n)
  &\mapsto  \left( y_1, \dotsc, y_{k-1}, \frac{1}{2} + \frac{y_k}{2}, \dotsc, y_k, \dotsc, y_n \right).
\end{alignat*}
Beide Abbildungen sind stetig, da sie in jeder Komponente stetig sind.
Nach der universellen Eigenschaft des Koprodukts gibt es eine eindeutige stetige Abbildung $\varphi \colon W_1 \amalg W_2 \to W$ mit $\varphi(x,1) = \varphi_1(x)$ und $\varphi(y,2) = \varphi_2(x)$ für alle $x \in W_1$ und $y \in W_2$.
(Wir notieren hier die Elemente von $W_1 \amalg W_2$ als Paare $(x,1)$ mit $x \in W_1$, bzw.\ $(y,2)$ mit $y \in W_2$.)
Man bemerke, dass $\varphi$ surjektiv ist.

Für die Äquivalenzrelation $\sim$ auf $W_1 \amalg W_2$ mit
\[
  x \sim y \iff \varphi(x) = \varphi(y)
  \quad
  \text{für alle $x, y \in W_1 \amalg W_2$}
\]
induziert die stetige Surjektion $\varphi$ nach der universellen Eigenschaft des Quotienten eine wohldefinierte stetige Bijektion
\[
          \induced{\varphi}
  \colon  (W_1 \amalg W_2)/{\sim}
  \to     W,
  \quad   \class{x}
  \mapsto \varphi(x).
\]
Da $W_1$ und $W_2$ kompakt sind, ist es auch $W_1 \amalg W_2$, und somit auch $(W_1 \amalg W_2)/{\sim}$.
Außerdem ist $W$ Hausdorff.
Somit ist die stetige Bijektion $\varphi$ nach dem üblichen Kompakt-Hausdorff-Argument bereits ein Homöomorphismus.

Man bemerke, dass die Äquivalenzrelation $\sim$ von
\[
        ( (x_1, \dotsc, x_{k-1}, 1, x_k, \dotsc, x_{n-1}), 1)
  \sim  ( (x_1, \dotsc, x_{k-1}, 0, x_k, \dotsc, x_{n-1}), 2)
\]
für alle $x = (x_1, \dotsc, x_{n-1}) \in [0,1]^{n-1}$ erzeugt wird, d.h.\ von
\[
  \left( i_0^k(x), 1 \right) \sim \left( i_1^k(x), 2 \right)
  \quad
  \text{für alle $x \in [0,1]^{n-1}$}.
\]
Wir erhalten nun eine Bijektion
\begin{align*}
              P^n X
  &\coloneqq  \left\{
                \text{$f \colon W \to X$ stetig}
              \right\}
  \\
  &\cong      \left\{
                \text{$f \colon (W_1 \amalg W_2)/{\sim} \to X$ stetig}
              \right\}
  \\
  &\cong      \left\{
                \text{$f \colon W_1 \amalg W_2 \to X$ stetig und $\sim$-equivariant}
              \right\}
  \\
  &=          \left\{
                \text{$f \colon W_1 \amalg W_2 \to X$ stetig
                      mit $f\left( ( i_1^k(x), 1) \right) = f\left( ( i_0^k(x), 2) \right)$ für alle $x \in [0,1]^{n-1}$}
              \right\}
  \\
  &\cong      \left\{
                (f, g)
               \suchthatscale
                \text{
                \begin{tabular}{c}
                  $f \colon W_1 \to X$, $g \colon W_2 \to X$ stetig mit
                \\
                  $f\left( i_1^k(x) \right) = g\left( i_0^k(x) \right)$ für alle $x \in [0,1]^{n-1}$
                \end{tabular}
                }
              \right\}
  \\
  &=          \left\{
                (f, g) \in P^n X \times P^n X
              \suchthatscale
                f \circ i_1^k = g \circ i_0^k
              \right\}
  \\
  &=          \left\{
                (f, g) \in P^n X \times P^n X
              \suchthatscale
                \left( i_1^k \right)^*(f) = \left( i_0^k \right)^*(g)
              \right\}
  =           P.
\end{align*}
Wir bezeichnen diesen Isomorphismus mit $\Phi$.
Dieser Isomorphismus liefert eine partiell definierte binäre Verknüpfung
\[
                      P^n X \times P^n X
  \supseteq           P
  \xrightarrow{\Phi}  P_n X
\]
mit
\[
          \Phi(f,g)
  \colon  [0,1]^n
  \to     X,
  \quad   (x_1, \dotsc, x_n)
  \mapsto \begin{cases}
            f(x_1, \dotsc, x_{k-1}, 2x_k, x_{k+1}, \dotsc, x_n)     & \text{falls $x_k \leq 1/2$},  \\
            g(x_1, \dotsc, x_{k-1}, 2x_k - 1, x_{k+1}, \dotsc, x_n) & \text{falls $x_k \geq 1/2$}.
          \end{cases}
\]

Im Fall $n = 1$ ist $P^1 X$ die Menge der stetigen Pfade in $X$.
Es ist $[0,1]^0 = \{*\}$ der einpunktige Raum, und für $f, g \in P^1 X$ gilt
\[
    (i_1^1)^*(f)(*)
  = f(i_1^1(*))
  = f(1)
  \quad\text{und}\quad
    (i_0^1)^*(g)(*)
  = g(i_0^1(*))
  = g(0).
\]
Deshalb gilt
\[
    P
  = \{
      (f,g) \in P^n X \times P^n X
    \suchthat
      (i_1^1)^*(f) = (i_0^1)^*(g)
    \}
  = \{
      (f,g) \in P^n X \times P^n X
    \suchthat
      f(1) = g(0)
    \}.
\]
Also besteht $P$ genau aus der Menge der konkatinierbaren Pfade.
Für $(f,g) \in P$ gilt außerdem
\[
    \Phi(f,g)(t)
  = \begin{cases}
      f(2t)   & \text{für $0 \leq t \leq 1/2$}, \\
      g(2t-1) & \text{für $1/2 \leq t \leq 1$}.
    \end{cases}
\]
Dies ist genau die Konkatination von Pfaden.




